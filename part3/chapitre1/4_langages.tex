\section{Langages et outils de développement}
Pour développer notre propre simulateur on utilise les outils et les langages de programmation suivant :

\subsection{Javascript}
JavaScript est un langage de programmation qui permet d’implémenter des mécanismes complexes sur une page web. À chaque fois qu’une page web fait plus que simplement afficher du contenu statique — afficher du contenu mis à jour à des temps déterminés, des cartes interactives, des animations 2D/3D, des menus vidéo défilants, etc... 
\\JavaScript a de bonnes chances d’être impliqué. C’est la troisième couche des technologies standards du web, les deux premières (HTML et CSS, Les trois couches se superposent naturellement. 
\begin{figure}[h]
	\centering
    \includegraphics[scale=0.1]{img/part3/4.1}
    \caption{Logo JavaScript}
\end{figure}

\subsection{VueJs}
Vue.JS, ou simplement Vue, est un framework progressif pour les interfaces utilisateur pour les apps et sites JavaScript. Il s’agit d’un des frameworks front-end JS les plus populaires. On le compare souvent à React, Angular, Ember, etc. Par leur approche et leur ressemblance, Vue et React partagent de nombreux points communs. Le framework apparaît à l’été 2013. Il est développé par Evan You. Peu à peu, Vue va faire parler de lui et s’imposer chez les développeurs JS.
\begin{figure}[h]
	\centering
    \includegraphics[scale=0.4]{img/part3/4.2}
    \caption{Logo VueJs}
\end{figure}

\subsection{NodesJs}
Node.js est une plateforme logicielle libre en JavaScript, orientée vers les applications réseau évènementielles hautement concurrentes qui doivent pouvoir monter en charge.

Elle utilise la machine virtuelle V8, la librairie libuv pour sa boucle d'évènements, et implémente sous licence MIT les spécifications CommonJS.

Parmi les modules natifs de Node.js, on retrouve http qui permet le développement de serveur HTTP. Ce qui autorise, lors du déploiement de sites internet et d'applications web développés avec Node.js, de ne pas installer et utiliser des serveurs webs tels que Nginx ou Apache.

Concrètement, Node.js est un environnement bas niveau permettant l’exécution de JavaScript côté serveur.

Node.js est utilisé notamment comme plateforme de serveur Web, elle est utilisée par GoDaddy, IBM, NetFlix, Amazon Web Services, Groupon3, Vivaldi, SAP4, LinkedIn5,6, Microsoft7,8, Yahoo!9, Walmart10, Rakuten, Sage et PayPal.
\begin{figure}[h]
	\centering
    \includegraphics[scale=0.4]{img/part3/4.3}
    \caption{Logo NodesJs}
\end{figure}

\subsection{Express Js}
ExpressJS est un framework qui se veut minimaliste. Très léger, il apporte peu de surcouches pour garder des performances optimales et une exécution rapide. Express ne fournit que des fonctionnalités d’application web (et mobile) fondamentales, mais celles-ci sont extrêmement robustes et ne prennent pas le dessus sur les fonctionnalités natives de NodeJS.
\begin{figure}[h]
	\centering
    \includegraphics[scale=0.1]{img/part3/4.4}
    \caption{Logo ExpressJs}
\end{figure} 

\newpage
\subsection{React Native}
React Native est un framework d'applications mobiles open source créé par Facebook. Il est utilisé pour développer des applications pour Android, iOS et UWP en permettant aux développeurs d’utiliser React avec les fonctionnalités natives de ces plateformes.

\begin{figure}[h]
	\centering
    \includegraphics[scale=0.1]{img/part3/4.5}
    \caption{Logo React Native}
\end{figure} 

\subsection{HTML5 et CSS3}
\textbf{HTML: }Le HyperText Markup Language, généralement abrégé HTML ou dans sa dernière version HTML5, est le langage de balisage conçu pour représenter les pages web. Ce langage permet : d’écrire de l’hypertexte, d’où son nom, de structurer sémantiquement la page, de mettre en forme le contenu, de créer des formulaires de saisie,

\textbf{CSS: } Les feuilles de style en cascade, généralement appelées CSS de l'anglais Cascading Style Sheets, forment un langage informatique qui décrit la présentation des documents HTML et XML. Les standards définissant CSS sont publiés par le World Wide Web Consortium

\begin{figure}[h]
	\centering
    \includegraphics[scale=0.3]{img/part3/4.6}
    \caption{Logo HTML et CSS}
\end{figure}