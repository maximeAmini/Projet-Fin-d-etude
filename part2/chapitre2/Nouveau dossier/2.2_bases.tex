\subsection{Les bases de données de Firebase}
La plateforme firebase, se compose de deux types de bases de données NoSQL qui sont Firebase Realtime Database et Firebase Firestore.

\subsubsection{1. Firebase Realtime Database:}

Est la base de données originale. Il s'agit d'une solution efficace a faible temps de latence pour les applications mobiles qui nécessitent des états synchronisés entre les clients en temps réel. Les données sont stockées sous forme de JSON \footnote{\textbf{JSON :} JavaScript Object Notation, format d'échange de données en texte lisible. Utilise pour représenter des structures de données.} et synchronisées en temps réel sur chaque client connectée.

Ce mode de stockage implique l'absence de tables et d'enregistrement, car les données sont représentées en tant que nœud dans l'arborescence JSON dans le cloud. Chaque noeud a un identifiant de clé unique qu'on peut nous même fournir ou laisser a Firebase le faire.

Pour la base de données Firebase Realtime Database nous allons parler de :

\begin{enumerate}

\item \textbf{Structuration de données :}

La construction d'une base de données nécessite une planification de la manière dont les données vont être sauvegardées puis récupérées pour rendre ce processus aussi simple que possible.
Toutes les données de la base de données Firebase Realtime sont stockées en tant qu'objects JSON. En effet quand on ajoute des données a l'arborescence JSON, cela devient un nœud dans la structure JSON existante avec une clé associée. Comme nous pouvons fournir nos propres clés, telles que des identiants d'utilisateur ou des noms sémantiques

\textit{\textbf {Par exemple :} considérons une application de discussion qui permet aux
utilisateurs de stocker un profil de base et une liste de contacts. Un profil utilisateur typique se trouve sur un chemin. L'utilisateur utilisateur 1 peut avoir une entrée de base de données qui ressemble a ceci :}

\item \textbf{Les caractéristiques de Realtime Database :}
\begin{itemize}
\item \textbf{Synchronisation en temps réel :} la base de données en temps réel de Firebase utilise la synchronisation des données. Ainsi, chaque fois que les données changent, tous les clients connectés sont mis a jour, aucune demande HTPP traditionnelle n'est nécessaire.
\item \textbf{Prise en charge automatique hors ligne :} le SDK offre la persistance du disque pour une synchronisation gratuite et automatique lorsque l'application est remise en ligne. Cela nous permet de ne plus se soucier de l'état hors connexion. En effet on peut ajouter, modifier et supprimer des données de manière transparente lorsqu'on est hors ligne, et le SDK se chargera de tout synchroniser correctement lorsque l'application sera de nouveau en ligne.
\end{itemize}

\item \textbf{Avantages de la base de données Firebase Realtime database :} 
Cette base de données comporte un ensembles d'avantages, dont nous citons
quelques uns :
\begin{itemize}
\item C'est une base de données NoSQL stockée dans le cloud de Google, donc disponible partout.
\item C'est une base de données NoSQL stockée dans le cloud de Google, donc disponible partout.
\item Mise a jour en temps réel sur plusieurs plates-formes et plusieurs utilisateurs.
\item ase de données multiplateforme.
\item Google héberge les données afin de ne plus se soucier du matériel ni d'autres choses.
\end{itemize}

\end{enumerate}

\subsubsection{Firebase Firestore :}
Est la nouvelle base de données phare de Firebase pour le développement d'applications mobiles. C'est aussi une base de données NoSQL, hébergée sur le cloud de Google (Firebase) et orientée Document. Les données collectées sont sauvegarder dans des documents qui sont organisées en collections. Chaque document contient un ensemble de paires clé-valeur. 
Pour mieux comprendre les base de données Firebase Firestore nous nous intéresserons aux :
\begin{enumerate}
\item \textbf{Les éléments de structuration des données :} 
La structuration des données dans Firestore se distingue par les 3 éléments suivants :

\begin{itemize}
\item \textbf{La donnée brute (DATA)}:représente la donnée qu'on souhaite sauvegarder dans une base de donnée, elle prend toutes les formes que la base restore peut supporter, par exemple un entier, une chaine de caractère, un tableau, etc.
\item \textbf{Le document (DOCUMENT)}: les données brute déjà générer, ne seront pas sauvegardées au hasard et n'importe ou, mais elles sont obligatoirement rattachées a un document, sous forme de propriétés/champs
\item \textbf{La collection (COLLECTION) :} Permet d'organiser sous forme de liste les documents. En effet, elle peut contenir un a plusieurs documents.
\end{itemize}
\item \textbf{Les types de données supportées :}
Le cloud Firestore prend en charge un certain nombre limité de types de données, et décrit aussi l'ordre de tri utilisé lors de la comparaison de valeurs du même type.
\item \textbf{Architecture des données dans restore:} 
Afin d'organiser les données, Firestore nous offre trois architectures possible pour mener a bien notre structure, a savoir les données imbriquées dans les documents, les sous-collections et les collections de niveau racine.

\begin{itemize}
\item \textbf{Les données imbriquées dans les documents :}
ou bien "Nested data in document" en anglais, est une architecture qui permet d'imbriquer des objets complexes tels que des tableaux ou des cartes dans des documents.
Facilite la configuration et la rationalisation de la structure des données si un utilisateur se dispose d'une liste simple et fixe et qu'il veut conserver ces données dans son document. En parallèle, si les données se développent avec le temps, la liste et le document augmentent eu aussi en croissance, ce qui permet d'avoir une latence dans la récupération des documents.

\textbf{Exemple:} Dans une application de discussion, on peut stocker les 3
dernières salles de discussion visitées par un utilisateur en tant que liste
imbriquée dans son profil.
\item \textbf{les Sous-collections:} ou bien "Subcollections" en anglais, est l'architecture qui permet a un utilisateur de créer des collections dans des documents lorsque il y a des données capable de se développer avec le temps. Son avantage est que si une liste s'allonge, la taille du document parent reste inchangée, aussi, elle offre des fonctionnalités de requête complètes sur les sous-groupes. Comme elle apporte un inconvénient sur la difficulté de supprimer un sous-groupe.
\item \textbf{les collections de niveau racine :} ou "Root-level collections" en anglais, est le type d'architecture qui offre la possibilité de créer des collections au niveau racine de la base de données pour organiser des ensembles de
données différentes. Parmi ses avantages on retrouve : une grande flexibilité et evolutivité, et des requêtes puissantes au sein de chaque collection. Mais cela ne l'empêche pas d'avoir des limites, a savoir la complexité d'obtenir des données naturellement hiérarchique si a base de données devient plus grande.

\textbf{Exemple:} Dans la même application de discussion, on peut créer une collection pour les utilisateurs et une autre pour les salles et les messages.
\end{itemize}

\end{enumerate}