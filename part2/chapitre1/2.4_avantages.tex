\subsection{Les avantages et les inconvénient}

\subsubsection{Avantages du compteur intelligent}
Le compteur intelligent offre plusieurs avantages aux consommateurs :
\begin{itemize}[label=\textbullet]
\item Les démarches sont facilitées 
\item Fini le passage d'un technicien pour le relevage du compteur, la consommation étant communiquée en temps réel au distributeur ;
\item Une consommation raisonnée et davantage écoresponsable ;
\item Une facturation suivie et dimensionnée quotidiennement grâce aux facilités de suivi, et donc plus précise.
\end{itemize}

En bref, le compteur intelligent permet un relevé automatique et précis à distance. En effet, avec ce type de compteur, le fournisseur peut commander vos appareils à distance et suivre toutes vos consommations.

\subsubsection{Ses inconvénients : des craintes plus ou moins fondées ?}
De nombreuses voix se sont opposées aux compteurs intelligent. Cependant, les raisons invoquées ne sont pas toutes avérées. Cette méfiance n'en est pas moins à prendre en compte, et révèle la nécessité de mettre en place des études et des contrôles réguliers et indépendants.

\begin{itemize}[label=\textbullet]
\item Première crainte, les radiofréquences CPL (courant porteur en ligne) injectées par les compteurs dans les câbles et appareils électriques. Ce type de technologie est également utilisé en domotique, pour les réseaux internet, etc. Certaines associations affirment que le CPL est dangereux pour la santé, mais 60 Millions de Consommateurs met en garde : pour le moment, aucune étude n'est parvenue à prouver cette dangerosité.
\item Deuxième point de méfiance : des pannes et des incendies ont déjà été observés sur les appareils en service. De fait, durant la phase test, 8 incendies ont été imputés non pas au compteur lui-même mais à des erreurs de manipulation pendant la pose (sur 300 000 installations). Des précisions qui rassurent quant à la capacité des installateurs à limiter ce risque.
\item Enfin, la surveillance généralisée des faits et gestes de la population est rendue possible par la transmission en temps réel des données de consommation, ce qui peut poser problème à certains. Néanmoins, c'est grâce à cette collecte d'informations que la domotique peut simplifier votre quotidien : mieux elle vous connaît, mieux elle peut agir pour vous. De plus, les données relevées par les compteurs intelligents sont surveillées par la CNIL, et soumises à des règles strictes.
\end{itemize}