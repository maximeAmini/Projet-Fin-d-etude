\subsection{Définition:}
Le terme anglais « Smart Grid » (en français : « réseau intelligent ») désigne un système de distribution d’énergie électrique qui adapte automatiquement, en autonomie, la production à la demande.

Pour parvenir à cet équilibre en flux tendu, le Smart Grid fait appel à un réseau de capteurs, et à des dispositifs de transmission et d’analyse informatique des données en temps réel (ainsi qu’au Big Data), qui intègrent et infléchissent les modes de production et de consommation de manière à parvenir à un résultat optimal en matière d’efficacité énergétique et de sécurisation.

En d'autre terme les Smart Grid sont des réseaux d’électricité qui, grâce à des technologies informatiques, ajustent les flux d’électricité entre fournisseurs et consommateurs, et cela en collectant des informations sur l’état du réseau ils contribuent à une adéquation entre production, distribution et consommation.

Il est nécessaire de différencier smart grid et compteur communicant (ou « smart meter »), qui renseigne le consommateur sur sa demande en électricité. « Smart grids » est une appellation générale pour l’ensemble des technologies et des infrastructures « intelligentes » installées. Chez le particulier, le compteur communicant est une première étape dans la mise en place des smart grids.

\subsubsection*{Caractéristiques Du Smart Grid:}

Les réseaux intelligents peuvent être définis selon quatre caractéristiques en matière de :
\begin{itemize}[label=\textbullet]
\item \textbf{flexibilité :} ils permettent de gérer plus finement l’équilibre entre production et consommation ;
\item \textbf{fiabilité :} ils améliorent l’efficacité et la sécurité des réseaux ;
\item \textbf{accessibilité :} ils favorisent l’intégration des sources d’énergies renouvelables sur l’ensemble du réseau ;
\item \textbf{économie :} ils apportent, grâce à une meilleure gestion du système, des économies d’énergie et une diminution des coûts (à la production comme à la consommation).
\end{itemize}

\subsubsection{Objectifs}

Les objectifs des smart grids sont multiples et répondent à différentes exigences de leur part:
\\Réduire l'impact du système électrique sur l'environnement.
\begin{itemize}[label=\textbullet]
\item Eduquer les utilisateurs et à les rendre plus actifs quant à leur consommation d'électricité, tout en leur permettant de la contrôler efficacement
\item Développer la production d'électricité décentralisée.
\item Garantir un faible coût, efficace et sans coupure de courant.
\item Permet de gérer facilement le système électrique et de faire face à la complexité
croissante du système électrique
\end{itemize}