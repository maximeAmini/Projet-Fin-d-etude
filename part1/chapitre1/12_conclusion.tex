\section*{Conclusion : }
A ce stade on peut dire que le Big Data est un écosystème large et complexe qui est né d’une réponse à un problème de temps de traitement des données.  Il nécessite la maitrise des  technologies matérielles et logicielles diverses (stockage, parallélisassions des  traitements, virtualisation,  …).  Le Big Data demande de la compétence et de l’expertise dans la maitrise et l’analyse des données. Les usages du Big Data sont très vastes qui touchent presque tous les secteurs d’activités (marketing, recherche, visualisation, …).
   
Les données collectées dans le cadre d’une étude Big Data peuvent avoir des origines et des formes très différentes, elles sont ensuite traitées à travers des algorithmes et des structures adaptés dans le but de produire une information. Aux fondements des algorithmes des solutions de Big Data se trouvent des lois de statistiques et de probabilités.
