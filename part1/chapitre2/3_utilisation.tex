\section{Utilisations du Cloud Computing}
Vous utilisez probablement en ce moment même le cloud computing sans le savoir. Si vous utilisez un service en ligne pour envoyer des courriers électroniques, modifier des documents, regarder des films ou regarder la télévision, jouer à des jeux ou stocker des images ou autres fichiers, il est probable que le cloud computing intervienne dans les coulisses. Les premiers services de cloud computing n’ont pas encore dix ans, mais un grand nombre d’organisations, par exemple des start-ups, des multinationales, des services administratifs ou des ONG, adopte cette technologie pour de nombreuses raisons.

Ici quelques exemples des possibilités d’utilisation des services cloud d’un fournisseur de cloud :

\begin{itemize}[label=\textbullet]
\item Créez des applications cloud natives Créez, déployez et mettez à l’échelle rapidement des applications (web, mobiles et API). Tirez parti des technologies et approches cloud natives, telles que les conteneurs, Kubernetes, l’architecture de microservices, la communication pilotée par des API et DevOps.
\item Tester et générer des applications Réduisez les coûts et délais de développement d’applications en utilisant des infrastructures cloud dont l’échelle peut être facilement adaptée.
\item Stocker, sauvegarder et récupérer des données Protégez vos données à moindre coût et à grande échelle en les transférant via Internet vers un système de stockage cloud hors site, accessible à partir de tout emplacement et appareil.
\item Analyser des données Unifiez vos données entre les équipes, les divisions et les emplacements dans le cloud. Utilisez ensuite des services cloud, par exemple de Machine Learning et d’intelligence artificielle, pour extraire des insights qui vous permettent de prendre des décisions éclairées.
\item Diffuser du contenu audio et vidéo Communiquez avec votre public en tout lieu, en tout temps et sur tout appareil via un système audio et vidéo haute définition mondialement distribué.
\item Incorporer de l’intelligence Utilisez des modèles intelligents pour interagir avec les clients et fournir des insights à partir des données capturées.
\item Diffuser des logiciels à la demande Également appelés logiciel en tant que service les logiciels à la demande vous permettent d’offrir à vos clients les versions et mises à jour les plus récentes des logiciels, à tout moment et en tout lieu.
\end{itemize}