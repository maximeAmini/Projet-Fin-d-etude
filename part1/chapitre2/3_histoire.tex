\section{L’histoire de l’IoT :}
C’est lors d’une présentation faite à Procter \& Gamble, en 1999, que Kevin Ashton, co-fondateur de l’Auto-ID Center au MIT, a mentionné pour la première fois l’Internet des Objets. Il souhaitait attirer l’attention des directeurs de P\& G sur les puces RFID (identification par radiofréquence). Cependant, il a nommé sa présentation « Internet of Things » pour intégrer la nouvelle tendance de l’année : Internet. Toutefois, l’idée d’appareils connectés existe depuis les années 1970. Ainsi, le premier objet connecté était une machine à Coca, à l’Université de Carnegie Mellon, au début des années 1980.

Via le Web, les développeurs pouvaient vérifier l’état de la machine, et déterminer si une boisson froide était disponible pour eux. L’IoT a ensuite évolué avec des technologies sans fil (Wifi par exemple), des MEMS (systèmes microélectromécaniques), des microservices et d’Internet. Ainsi, la technologie opérationnelle (OT – Operational Technology) et la technologie de l’information (IT) se sont rapprochées. Ceci a permis d’analyser des données non structurées, générées par des machines, pour en tirer des axes d’amélioration.

L’Internet des Objets utilise comme base la connectivité M2M (Machine to Machine). C’est-à-dire que des machines se connectent entre elles, via un réseau, sans interaction humaine. L’IoT est donc un réseau de milliards de noeuds (capteurs ici), qui connectent des personnes, des systèmes et d’autres applications pour collecter et partager des données. Néanmoins, le concept d’écosystème de l’Internet of Things ne s’est concrétisé qu’au milieu des années 2010. Une avancée que l’on doit au gouvernement chinois, qui a déclaré qu’il ferait de l’IoT une priorité stratégique, notamment dans sa stratégie de reconnaissance faciale et de fichage du peuple chinois.
