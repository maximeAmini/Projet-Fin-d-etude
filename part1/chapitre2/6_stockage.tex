\section{Types et modes de stockage dans le cloud computing :}
\subsection{Les types de stockage dans le cloud computing : }
Tous les clouds ne sont pas identiques et aucun type de cloud computing ne convient à tout le monde. Plusieurs modèles, types et services différents ont évolué pour nous aider à trouver la solution adaptée à vos besoins

\begin{itemize}[label=\textbullet]
\item \textbf{Cloud public :} est détenu et exploité par un fournisseur de cloud tiers, qui propose des ressources de calcul, telles que des serveurs et du stockage, via Internet. Microsoft Azure est un exemple de cloud public. Dans ce dernier, tout le matériel, tous les logiciels et toute l’infrastructure sont la propriété du fournisseur du cloud. Vous accédez à ces services et vous gérez votre compte par l’intermédiaire d’un navigateur web. 
\item \textbf{Cloud privé :} est l’ensemble des ressources de cloud computing utilisées de façon exclusive par une entreprise ou une organisation. Le cloud privé peut se trouver physiquement dans le centre de données local des entreprises, dans les quelles paient également des fournisseurs de services pour qu’ils hébergent leur cloud privé qui est un cloud dans lequel les services et l’infrastructure se trouvent sur un réseau privé. 
\item \textbf{Cloud hybride:} regroupe des clouds publics et privés, liés par une technologie leur permettant de partager des données et des applications. En permettant que les données et applications se déplacent entre des clouds privé et public, un cloud hybride offre à l’entreprise une plus grande flexibilité, davantage d’options de déploiement et une optimisation de l’infrastructure, de sécurité et de conformité existantes.
\item \textbf{Multicloud :} est un type de déploiement cloud qui implique l'utilisation de plusieurs clouds publics. Autrement dit, une organisation disposant d’un déploiement multi-cloud loue des serveurs et des services virtuels auprès de plusieurs fournisseurs externes. Les déploiements multi-cloud peuvent aussi être des clouds hybrides, et vice-versa.
\end{itemize}

\subsection{Les formats de stockage dans le cloud : }
Il existe trois formats de stockage de données dans le cloud :

\begin{itemize}[label=\textbullet]
\item \textbf{Stockage en mode bloc :} Le stockage en mode bloc permet de diviser un seul volume de stockage (par exemple, un nœud de stockage dans le cloud) en plusieurs instances individuelles appelées blocs. Il s'agit d'une solution de stockage rapide à faible latence, idéale pour les charges de travail hautes performances.
\item \textbf{Stockage en mode objet :} Le stockage en mode objet implique d'attribuer à chaque donnée des identifiants uniques, que l'on appelle les métadonnées. Compte tenu du fait que les objets ne sont ni compensés ni chiffrés, ils sont rapidement accessibles à très grande échelle. Il s'agit donc d'une solution idéale pour les applications cloud-native.
\item \textbf{Stockage en mode fichier :} Le stockage en mode fichier est le plus utilisé sur les systèmes NAS. Il permet d'organiser les données et de les présenter aux utilisateurs. Sa structure hiérarchique permet de parcourir les données du haut vers le bas en toute simplicité, mais augmente le temps de traitement.
\end{itemize}