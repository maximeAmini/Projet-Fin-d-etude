\section{Le future de l'iot ? (Internet of Everything)}
L’Internet of Everything(IOE) (L’Internet du Tout) concerne et regroupe les connexions entre des personnes, des objets, des données et des processus pour former un système commun interdépendant dont le but est d’améliorer les expériences et de prendre des décisions plus éclairées.

Bien que le concept apparaisse comme un développement naturel du IoT, il englobe le concept plus large de connectivité : la philosophie de l’IoE décrit le monde dans lequel des milliards de capteurs sont implantés dans des milliards d’appareils, de machines et d’objets ordinaires, tous connectés sur des réseaux publics ou privés à l’aide de protocoles standard et propriétaires, ce qui leur confère de nombreuses opportunités de mise en réseau, les rendant ainsi plus intelligents.

Pour les entreprises, autant que pour les gouvernements et les particuliers, l’objectif principal de la technologie IoE est de convertir les informations collectées en actions, de faciliter la prise de décision basée sur des données.

\subsection{Caractéristiques de l'IOE :}
\begin{itemize}[label=\textbullet]
\item \textbf{Décentralisation :} Les données sont traitées non pas dans un seul centre, mais dans de nombreux noeuds distribués
\item \textbf{Entrée et sortie de données :} des données externes peuvent être injectées dans des périphériques et redistribuées au réseau.
\item \textbf{Relation avec toutes les technologies de transformation numérique :} Cloud computing, IA, Machine Leaning, IoT, Big Data, etc.
\end{itemize}

\subsection{Éléments constitutifs de l’IoE:}
\begin{enumerate}
\item \textbf{Personnes :} Considérées comme des noeuds finaux connectés sur Internet pour partager des informations et des activités, les personnes communiquent leurs points de vue personnels via des sites Web, des applications ou des appareils connectés qu’ils utilisent.

Ces données sont ensuite utilisées par des algorithmes d’intelligence artificielle et d’autres technologies intelligentes analysent ces données pour «comprendre» les problèmes humains et fournir un contenu pertinent en fonction de leurs besoins personnels ou professionnels.

\item \textbf{Les objets:} Ici, on parle de tous les objets connectés (capteurs physiques, dispositifs, actionneurs et autres éléments générant des données ou recevant des informations d’autres sources), donc le concept IoT pur.
\item \textbf{Les données:} Les données brutes générées par les périphériques et ensuite analysées puis traitées en informations, utiles pour permettre des décisions intelligentes et des mécanismes de contrôle; elles permettent d’autonomiser des solutions intelligentes, telle que l’évaluation des besoins en refroidissement d’une pièce, sur base de journaux de température précédents.
\item \textbf{Les processus:} Différents processus basés sur l’intelligence artificielle, l’apprentissage automatique, les réseaux sociaux ou d’autres technologies garantissent que les bonnes informations sont envoyées à la bonne personne au bon moment.

Le but des processus est de garantir la meilleure utilisation possible du Big Data. On peut citer ici par exemple, l’utilisation d’appareils de fitness intelligents et de réseaux sociaux pour annoncer des offres de soins de santé pertinentes à des clients potentiels.
\end{enumerate}

\subsection{La difference entre l’IOT et l’IOE :}
Les deux concepts d’IoT et d’IoE ne sont pas antagonistes. Bien au contraire, ils se complètent. L’Internet of Everything doit plutôt être considéré comme l’extension naturelle, comme la suite logique de l’Internet des objets. Rappelons qu’Internet est apparu par vagues successives. La première vague a consisté à connecter les ordinateurs (des militaires et scientifiques) entre eux avec le protocole IP. Puis la seconde vague a permis de connecter le grand public grâce à l’apparition du web, des réseaux mobiles, des réseaux sociaux, et à la démocratisation des PC personnels et des smartphones. La troisième vague est celle de l’IoT, qui vise à connecter tous les objets pouvant contenir un capteur. Au-delà des hommes et des ordinateurs, l’Internet des objets offre la possibilité inédite de connecter le monde physique dans son ensemble. Lorsque "tout" sera connecté, alors l’Internet of Everything sera devenu une réalité