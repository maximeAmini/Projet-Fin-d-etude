\section{Avantages et Limites:}

\subsection{Les principaux avantages:}

Cassandra est utilisée par certaines des plus grandes entreprises du monde telles que Facebook, Netflix, Twitter, Cisco, eBay… voici certains des avantages qui lui permettent de se distinguer de la concurrence.

\begin{itemize}[label=\textbullet]
	\item Tout d’abord, Apache Cassandra est capable de prendre en charge les données structurées, non structurées ou semi-structurées. Elle est également capable de supporter les changements dynamiques apportés aux structures de données pour s’adapter aux besoins changeants.
	
	\item Un autre avantage est son architecture scalable de façon linéaire. Il suffit d’ajouter de noeuds pour l’adapter à une hausse de la demande. En outre, les données peuvent être distribuées de façon homogène sur de multiples centres de données par le biais d’un processus de réplication des données.
	
	\item Cette base de données est également très fiable, car les éventuelles défaillances de noeuds n’affectent par les performances générales. Cassandra se distingue aussi par son impressionnante vitesse d’écriture de données.
\end{itemize}

\subsection{Les limites:}
La principale limitation concernant les tailles des colonnes et des super-colonnes, toutes les données pour une valeur de clé, doivent tenir sur le disque d’une seule machine. Parce que la valeur des clés seules détermine les noeuds responsable de la réplication des données, la quantité de données associées à une clé a cette limitation.